\documentclass[12pt]{book}
\author{Andr\'{e} Kirst}
\date{\today{}, Leipzig}
\title{School - Projektdokumentation}
\usepackage{geometry}
\geometry{a4paper, top=25mm, left=35mm, right=25mm, bottom=25mm}
\usepackage[ngerman]{babel} %deutsche Silbentrennung
\usepackage{graphicx}
\usepackage{titlesec}
\usepackage{fancyhdr} %Paket laden
\usepackage{caption, booktabs}
\usepackage[xindy]{glossaries}	
\usepackage{tikz}
\usepackage{ifthen}
\usepackage{xstring}
\usepackage{calc}
\usepackage{pgfopts}
\usepackage{pgfkeys}

\usepackage{listings}
\usepackage{xcolor}

\lstdefinestyle{sharpc}{language=[Sharp]C, frame=lr, rulecolor=\color{blue!80!black}}

\begin{document}

\pagestyle{fancy} %eigener Seitenstil
\fancyhf{} %alle Kopf- und Fußzeilenfelder bereinigen
\fancyhead[L]{School} %Kopfzeile links
\fancyhead[C]{} %zentrierte Kopfzeile
\fancyhead[R]{} %Kopfzeile rechts
\renewcommand{\headrulewidth}{0.4pt} %obere Trennlinie
\fancyfoot[C]{\thepage} %Seitennummer
\renewcommand{\footrulewidth}{0.4pt} %untere Trennlinie

\fancyhead[OR]{} % "O" steht für "odd", also ungerade Seiten
\fancyhead[ER]{} % "E" für "even", also gerade Seiten.


\maketitle

\section{Einleitung}
\label{sec:Einleitung}
asd

\newpage

\tableofcontents


\documentclass[11pt]{book}
\author{André Kirst}
\date{\today{}, Leipzig}
\title{School - Projektdokumentation}
\usepackage{geometry}
\geometry{a4paper, top=25mm, left=35mm, right=25mm, bottom=25mm}
\usepackage[ngerman]{babel} %deutsche Silbentrennung
\usepackage{titlesec}

\begin{document}

School!

\section{Person}
\label{sec:Person}
\subsection{Fachliche Beschreibung}
\label{sec:Person_Fachliche_Beschreibung}
\subsection{Attribute}
\label{sec:Person_Attribute}
\subsection{Beziehungen}
\label{sec:Person_Beziehungen}

\end{document}


\chapter{Technischer Aufbau}

\printglossary[style=altlist,title=Glossar]

%Abkürzungen ausgeben
\deftranslation[to=German]{Acronyms}{Abkürzungsverzeichnis}
\printglossary[type=\acronymtype,style=long]

\end{document}
